\documentclass[12pt]{article}

% Basic stuff
\usepackage{fullpage}
\usepackage{graphics}
\usepackage[parfill]{parskip}
\usepackage[utf8]{inputenc}

% Skip one line between paragraphs
\parskip1em
 
% Standard things to include for math   
\usepackage{amsmath,amssymb,amsfonts,amsthm}

% Other stuff
\usepackage{hyperref}
\usepackage{datetime} \usdate
\usepackage{enumitem}


% Some of Ebrahim's definitions
\newcommand{\note}[1]{[#1]}
\newcommand{\done}{\\\hspace*{0pt}\hfill$\blacksquare$}
\def\N{\mathbb{N}}
\def\R{\mathbb{R}}
\def\Q{\mathbb{Q}}
\def\Z{\mathbb{Z}}
\def\e{\varepsilon}
\def\d{\delta}
\newcommand{\seq}[1]{\left(#1\right)_{n\in\N}}
\newcommand{\settext}[2]{\left\{ #1\ |\ \text{#2} \right\}}
\renewcommand{\emptyset}{\{\hspace{-2pt}\}}
\newcommand{\powerset}[1]{\mathcal{P}(#1)}
\newcommand{\dom}[1]{\operatorname{dom}(#1)}
\newcommand{\ran}[1]{\operatorname{ran}(#1)}
\newcommand{\rev}[1]{(#1)^{\dashv}}
\newcommand{\injrightarrow}{\xrightarrow{\text{inj}}}
\newcommand{\surjrightarrow}{\xrightarrow{\text{surj}}}
\newcommand{\bijrightarrow}{\xrightarrow{\text{bij}}}
\newcommand{\AND}{\wedge}
\newcommand{\OR}{\vee}
\newcommand{\ex}[1]{\paragraph{Exercise:}#1}



\begin{document}

\begin{center} {\Large \scshape Mathematical Proofs and Mathematical Writing}
\end{center}
(compiled on \today\ at\ \currenttime)
\hfill

\tableofcontents

\section{Introduction}

\note{what to put here:
this course is about both set theory and writing.
set theory is axiomatic. describe what i mean by that, why we would do that, and the historical relevance of it.
why rigor anyway?
logic can also be done this way, but we will not.
we will follow a more intuitive and informal writing-focused approach for logic,
saving our rigor-enegery for set theory.
perhaps include a discussion levels of rigor and formality, with one level relegating the rigor to other levels.
mention the importance of writing and the roughness of this transition for many people.
mention the value of forgetting what you knew before, but not really forgetting it all.}

\note{Write the general intro later, maybe last.}



\section{Mathematical Language}
\label{sec:logic}

This section will develop mathematical logic and show you how logical arguments factor into writing.
We do not yet have anything to argue \emph{about}, so our ``proofs'' will be about nothing in particular.
Section \ref{sec:sets} will later give some mathematical objects to talk about, and we will then put
content into the empty skeleton arguments of this section.

\subsection{Parts of Speech}

In a language, a \emph{part of speech} is a collection of words that share grammatical properties.
For example here are some common English parts of speech: pronouns, nouns, verbs, adjectives, and adverbs.
Different languages can have different parts of speech that are useful for describing their grammar.
To build up the grammar of a language, one can specify how parts of speech fit together to form larger
\emph{constituents}, such as noun phrases and clauses. One can further describe how
constituents fit together to form even more complex constituents, eventually grammatical sentences.

Mathematics is a language, and so it makes sense to build up its grammar (its \emph{syntax}) using this same strategy.
Instead of pronouns, nouns, verbs, noun phrases, sentences, and so on, there are just four constitutuents in
mathematics. Two are basic parts of speech and two are more complex constituents:
\begin{itemize}
\item \emph{Variables} are a basic part of speech.
\item \emph{Constants} are a basic part of speech.
\item \emph{Terms} can be built out of variables, constants, other terms, and propositions.
\item \emph{Propositions} can be built out of variables, constants, terms, and other propositions.
\end{itemize}

Let's go through each of these in detail.

\def\sp{\hspace{1em}}
\paragraph{Variables}
Variables serve as placeholders, waiting to be replaced by other symbols.
They are most like pronouns in English.
For example the pronoun ``it'' is a placeholder referring to something in a discussion, but what it refers to depends on the context of the discussion.
Variables are like that.
Here are some variables:
\begin{center}
$a$ \sp $b$ \sp $A$ \sp $f$ \sp $q$ \sp $x$\sp $\alpha$\sp $\beta$
\end{center}
Note that we are often talking about elements of \emph{language}, and not about what those elements \emph{refer} to.
Germany is not a noun, it's a country. ``Germany'' is, however, a noun. Similarly, we can say that $x$ is not a variable, but rather ``$x$'' is.
Quotation marks can help us make the distinction between a symbol and what the symbol refers to.
As symbols, variables are copies of letters coming from some alphabet, written in ink or pixels.


\def\sp{\hspace{1em}}
\paragraph{Constants}
Constants are used as fixed names for specific mathematical objects.
They are most like \emph{proper} nouns in English.
Here are some constants:
\begin{center}
$0$ \sp $1$ \sp $2$ \sp $3$ \sp $4$ \sp $5$ \sp $6$ \sp $7$ \sp $8$ \sp $9$ \sp $\N$ \sp $\R$ \sp $\emptyset$
\end{center}


\def\sp{\hspace{1em}}
\paragraph{Terms}
Terms are strings of marks (expressions) that refer to \emph{mathematical objects}.
Since variables and constants refer to mathematical objects, they are special cases of terms.
But you can also have more complicated terms that are built out of simpler ones.
Terms are most like \emph{noun phrases} in English.
For example the phrase ``the cup that you drank from'' is a noun phrase;
it doesn't make any assertion but rather it just refers to a \emph{thing}.
Here are some terms:
\begin{itemize}[label=\sp]
\item $x$
\item $7$
\item $\{2,3,a,7\}$
\item $f(x)$
\item $\{(z,w)\} \circ g$
\item $S\times Q$
\item the square of the seventh prime number
\item a triangle
\item twelve dimensional euclidean space
\item the collection of even integers
\item $\settext{n}{$n\in\Z$ and $n$ is even}$
\item $\settext{n}{$(n\in\Z)\AND (\exists k\,|\,(k\in\Z)\AND(2k=n))$}$
\end{itemize}
The last three terms actually refer to the same mathematical object; this will soon become clear
when we get into \emph{how} the symbols refer to objects (semantics). For now we are only looking at
\emph{which} strings of symbols \emph{can} refer to objects (syntax).
You can see that some terms are more ``symbolic,'' while others are more ``Englishy.''
The formal language of mathematics is purely symbolic,
but we almost never use the language in its purest form.
Typically, we communicate by some combination of English and mathematics.


\def\sp{\hspace{1em}}
\paragraph{Propositions}
Propositions are strings of marks (expressions) that \emph{make assertions}.
They are most like \emph{sentences} in English (sentences in the indicative mood).
Propositions can be true or false. 
Here are some propositions:
\begin{itemize}[label=\sp]
\item $x\in S$.
\item $5\notin x$.
\item $0=2$.
\item $(x\neq y) \AND (x\neq z) \AND (y\neq z)$.
\item $x$, $y$, and $z$ are distinct.
\item Either $A\subseteq P$ or $x\in S$, but not both.
\item $f:X\rightarrow Y$.
\item If $x\in S$, then we either have $x\notin W$ or we have $x\in\Z$.
\item $\neg(X\times Y \subseteq Z)$.
\item Every integer is even.
\item $(\forall n \, | \, (n\in\Z)\Rightarrow (\exists k\,|\,(k\in\Z)\AND(2k=n)))$.
\end{itemize}
The last two propositions are actually saying the same thing, as we will see when we get into semantics.
Again, you can see that some propositions are more ``symbolic,'' while others are more ``Englishy.''
Typically, we make mathematical assertions by using some combination of English and mathematics.
The English that we use is a crude, but human-friendly, stand-in for formal mathematical statements.



\ex{
Determine whether each of the following is a term or a proposition.
\begin{enumerate}
\item $n$
\item $1+1=0$
\item $n$ is an odd integer
\item an odd integer
\item $f$ is a function with domain $S$
\item $1+(2+3)$
\item the empty set
\item the sum of two vectors is another vector
\item the zero vector
\item the evenness of $2$ % this one is a trick, it's neither unless you are doing metamathematics
\end{enumerate}
}

\subsection{Theorems and Proofs}

% in the end it's all about the propositions. terms are only useful because they can be used to make proposiitons
% propositions are supposed to SAY things, though we haven't given them meaning yet
% explain what is an axiom and what is aproof and what is a theorem
% look at an example of a formal proof (one that you can also write informally soon)
% math is for HUMANS and so, while nice, this is just not where we want to be
% we want some kind of sweet spot where we have the backing of the formal development, but we are still communicating proofs to each other in efficient (though crude and vague) human language
% and so we want informal proofs. show how the previous example looks as an informal proof
% now you can see what this course is about: while it is partly giving you some content like set theory, it's mainly about how to communicate proofs in that sweet spot

\subsection{Logic and Writing Mathematical Arguments}

% run through the rules and some useful derived rules of logic
% the focus is not formal derivation, but rather to make the rules intuitive and to examine how they influence *writing*
% where truth tables can help make rules intuitive, bring them in. otherwise truth tables are not a main focus


% a comment (maybe put this earlier) on how repetitiveness is okay for now, and better to fix it later. perhaps sian comments on how fix


\section{Set Theory} % and still writing though...
\label{sec:sets}


% develop set theory along the lines of jacoby notes

% some milestones to reach: relations, functions, induction, recursion, natural numbers

% appendix: put some sample well written proofs and badly written proofs




\end{document}



